% Do not change document class, margins, fonts, etc.
\documentclass[a4paper,oneside,bibliography=totoc]{scrbook}

% some useful packages (add more as needed)
\usepackage[utf8]{inputenc}
\usepackage{graphicx}
\usepackage{latexsym}
\usepackage{amsmath}
\usepackage{amssymb}
\usepackage{tabularx}
\usepackage{booktabs}
\usepackage{algorithm} % you can modify the algorithm style to your liking
\counterwithin{algorithm}{chapter} % so that algorithms have chapter numbers as well
\usepackage{algorithmic}
\usepackage{csquotes}
\renewcommand{\algorithmiccomment}[1]{\hfill\textit{// #1}}
\usepackage[usenames,dvipsnames]{xcolor}
\usepackage[colorlinks,citecolor=Green]{hyperref} % you may change/remove the colors
\usepackage{lipsum} % you do not need this

% chicago citation style
\usepackage{natbib}
\bibliographystyle{chicagoa}
\setcitestyle{authoryear,round,semicolon,aysep={},yysep={,}} \let\cite\citep

% example enviroments (add more as needed)
\newtheorem{definition}{Definition} \newtheorem{proposition}{Proposition}

\begin{document}

\frontmatter \subject{Master Thesis} % change to appropriate type
\title{Title of Your Master Thesis}
\author{Max Muster\\
  (matriculation number 12345678)} \date{July 1, 2024}
\publishers{{\small Submitted to}\\
  Data and Web Science Group\\
  Prof.\ Dr.\ Right-Name-Here\\
  University of Mannheim\\}
\maketitle

\chapter{Abstract}

Your thesis must contain an abstract. A good reference for thesis writing
is~\citet{zobel2014}; we highly recommend that you study this or a similar book
during your studies. He writes the following about the abstract:

\blockcquote{zobel2004}{%
  An abstract is typically a single paragraph of about 50 to 200 words. The
  function of an abstract is to allow readers to judge whether or not the paper
  is of relevance to them. It should therefore be a concise summary of the
  paper's aims, scope, and conclusions. There is no space for unnecessary text;
  an abstract should be kept to as few words as possible while remaining clear
  and informative. Irrelevancies, such as minor details or a description of the
  structure of the paper, are inappropriate, as are acronyms, abbreviations, and
  mathematics. Sentences such as "We review relevant literature" should be
  omitted.

  The more specific an abstract is, the more interesting it is likely to be.
  Instead of writing "space requirements can be significantly reduced", write
  "space requirements can be reduced by 60\%". Instead of writing "we have a new
  inversion algorithm", write "we have a new inversion algorithm, based on
  move­to-front lists".

  Many scientists browse research papers outside their area of expertise. You
  should not assume that all likely readers will be specialists in the topic of
  their paper-abstracts should be self-contained and written for as broad a
  readership as possible. Only in rare circumstances should an abstract cite
  another paper (for example, when one paper consists entirely of analysis of
  results in another), in which case the reference should be given in full, not
  as a citation to the bibliography.}

% table of contents
\begingroup%
\hypersetup{hidelinks}% disable link color in TOC only
\tableofcontents%
\endgroup

% none of the things below is needed, but you may add them if you feel that they
% are helpful for your work \listofalgorithms \listoffigures \listoftables
% \listtheorems{definition} \listtheorems{proposition}


% okay, start new numbering ... here is where it really starts
\mainmatter

\chapter{Introduction}
\label{ch:intro}

The structure of this example thesis is \emph{non-binding}, but is highly
topic-dependent and should be discussed with your advisor. You probably do not
want to go below the level of subsections, though.

To cite, do it in one of the following ways:
\begin{itemize}
\item A single paper in text: As shown by \citet{doe2024proof}, X holds.
\item Same with additional information: As shown by \citet[p. 20]{doe2024proof},
  X holds.
\item A single paper in parenthesis: X holds \cite{doe2024proof}.
\item Multiple papers in parenthesis: X holds
  \cite{brown2022techreport,smith2023conference,lee2023journal,doe2024proof}.
\item Use footnotes to reference a website, such as the DWS thesis
  guidelines.\footnote{\url{https://www.uni-mannheim.de/dws/teaching/thesis-guidelines/}}
  Note that footnote marks are set after punctuation in English text. Only
  reference a website when needed---e.g., the website of a library or a blog
  post---and always cite corresponding scientific publications (if any) in
  addition.
\end{itemize}
Format the bibliography consistently, e.g., as in the the example bibliography
in this template. Only cite archival versions of papers (such as those published
on \href{https://arxiv.org}{arXiv}) when there are no corresponding official
publications (such as in conference proceedings); if there are such
publications, then cite those instead.


Some examples of references are:
\begin{enumerate}
\item Table~\ref{tab:sample} is an example table.
\item Section~\ref{sec:preliminaries} is a later section of this thesis.
\item Algorithm~\ref{alg:proofX} is an example algorithm.
\item And finally, Equation~\eqref{eq:s} is an example equation.
\end{enumerate}

Here is the equation referred to above:
\begin{equation}
  \label{eq:s}
  \mathcal{S}=\{\,a\cdot \operatorname{relu}(\mathbf{W^\top}\mathbf{x}) :
  a\in\mathbb{R} \,\}
\end{equation}

\textbf{Important:} Fill out and sign the last page when you submit/deliver the
final version of your thesis. Otherwise, your work cannot be accepted for legal
reasons.

The purpose of the introduction as summarized by~\citet{zobel2004}:

\blockcquote{zobel2004}{%
  An introduction can be regarded as an expanded version of the abstract. It
  should describe the paper's topic, the problem being studied, references to
  key papers, the approach to the solution, the scope and limitations of the
  solution, and the outcomes. There needs to be enough detail to allow readers
  to decide whether or not they need to read further. It should include
  motivation: the introduction should explain why the problem is interesting,
  what the relevant scientific issues are, and why the solution is a good one.

  That is, the introduction should show that the paper is worth reading and it
  should allow the reader to understand your perspective, so that the reader and
  you can proceed on a basis of common understanding.

  Many introductions follow a five-element organization:
  \begin{enumerate}
  \item A general statement introducing the broad research area of the
    particular topic being investigated.
  \item An explanation of the specific problem (difficulty, obstacle, challenge)
    to be solved.
  \item A brief review of existing or standard solutions to this problem and
    their limitations.
  \item An outline of the proposed new solution.
  \item A summary of how the solution was evaluated and what the outcomes of the
    evaluation were.
  \end{enumerate}

  An interesting exercise is to read other papers, analyze their introductions
  to see if they have this form, and then decide whether they are effective. The
  introduction can discuss the importance or ramifications of the conclusions
  but should omit supporting evidence, which the interested reader can find in
  the body of the paper. Relevant literature can be cited in the introduction,
  but unnecessary jargon, complex mathematics, and in-depth discussion of the
  literature belong elsewhere.

  A paper isn't a story in which results are kept secret until a surprise
  ending. The introduction should clearly tell the reader what in the paper is
  new and what the outcomes are. There may still be a little suspense: revealing
  what the results are does not necessarily reveal how they were achieved. If,
  however, the existence of results is concealed until later on, the reader
  might assume there are no results and discard the paper as worthless.}

\chapter{Literature Review}
\label{ch:related_work}

\citet{zobel2004} writes:

\blockcquote{zobel2004}{%
  Few results or experiments are entirely new. Most often they are extensions of
  or corrections to previous research-that is, most results are an incremental
  addition to existing knowledge. A literature review, or survey, is used to
  compare the new results to similar previously published results, to describe
  existing knowledge, and to explain how it is extended by the new results. A
  survey can also help a reader who is not expert in the field to understand the
  paper and may point to standard references such as texts or survey articles.

  In an ideal paper, the literature review is as interesting and thorough as the
  description of the paper's contribution. There is great value for the reader
  in a precise analysis of previous work that explains, for example, how
  existing methods differ from one another and what their respective strengths
  and weaknesses are. Such a review also creates a specific expectation of what
  the contribution of the paper should be-it shapes what the readers expect of
  your work, and thus shapes how they will respond to your ideas.

  The literature review can be early in a paper, to describe the context of the
  work, and might in that case be part of the introduction; or the literature
  review can follow or be part of the main body, at which point a detailed
  comparison between the old and the new can be made. If the literature review
  is late in a paper, it is easier to present the surveyed results in a
  consistent terminology, even when the cited papers have differing nomenclature
  and notation. In many papers the literature review material is not gathered
  into a single section, but is discussed where it is used-background material
  in the introduction, analysis of other researchers' work as new results are
  introduced, and so on. This approach can help you to write the paper as a
  flowing narrative.

  An issue that is difficult in some research is the relationship between new
  scientific results and proprietary commercial technology. It often is the case
  that scientists investigate problems that appear to be solved or addressed in
  commercial products. For example, there is ongoing academic research into
  methods for information retrieval despite the success of the search engines
  deployed on the web. From the perspective of high research principle, the
  existence of a commercial product is irrelevant: the ideas are not in the
  public domain, it is not known how the problems were solved in the product,
  and the researcher's contribution is valid. However, it may well be reckless
  to ignore the product; it should be cited and discussed, while noting, for
  example, that the methods and effectiveness of the commercial solution are
  unknown. }

An example structure for the literature review is given below; as before, this
structure is non-binding and should be changed as appropriate.

\section{Preliminaries}
\label{sec:preliminaries}

\section{Related Work on A}
\label{sec:related_work_A}

\section{Related Work on B}

\section{Related Work on C}

\section{Summary}

\chapter{Body Chapter 1}
\citet{zobel2004} writes:

\blockcquote{zobel2004}{%
  The body of a paper should present the results. The presentation should
  provide necessary background and terminology, explain the chain of reasoning
  that leads to the conclusions, provide the details of central proofs,
  summarize any experimental outcomes, and state in detail the conclusions
  outlined in the introduction. Descriptions of experiments should permit
  reproduction and verification, as discussed in Chapter 11. There should also
  be careful definitions of the hypothesis and major concepts, even those
  described informally in the introduction. The structure should be evident in
  the section headings. Since the body can be long, narrative flow and a clear
  logical structure are essential.

  The body should be reasonably independent of other papers. If, to understand
  your paper, the reader must find specialized literature such as your earlier
  papers or an obscure paper by your advisor, then its audience will be limited.

  In some disciplines, research papers have highly standardized structures.
  Editors may require, for example, that you use only the four headings
  Introduction-Methods-Results-Discussion. This convention has not taken hold in
  computer science, and in some cases such a structure impedes a clear
  explanation of the work. For example, use of fixed headings may prohibit
  development of a complex explanation in stages. In work combining two query
  resolution techniques, we had to determine how they would interact, based on a
  fresh evaluation of how they behaved independently. The final structure was,
  in effect,
  Introduction-Background-Methods-Results-Discussion-Methods-Results­Discussion.

  Even if the standardized section names are not used, the body needs these
  elements, if not necessarily under their standard headings. Components of the
  body might include, among other things, background, previous work, proposals,
  experimental design, analysis, results, and discussion. Specific research
  projects suggest specific headings. For the "compression for fast external
  sorting" project sketched earlier, the complete set of section headings might
  be:

  \begin{enumerate}
  \item Introduction
  \item External sorting
  \item Compression techniques for database systems
  \item Sorting with compression
  \item Experimental setup
  \item Results and discussion
  \item Conclusions
  \end{enumerate}

  The wording of these headings does not follow the standard form, but the
  intent of the wording is the same. Sections 2 and 3 are the background;
  Section 4 contains novel algorithms, and Sections 4 and 5 together are the
  methods.

  The background material can be entirely separate from the discussion of
  previous work on the same problem. The former is the knowledge the reader
  needs to understand your contribution. The latter is, often, alternative
  solutions that are superseded by your work. Together, the discussion of
  background and previous work also introduce the state of the art and its
  failings, the importance and circumstances of the research question, and
  benchmarks or baselines that the new work should be compared to.

  A body that consists of descriptions of algorithms followed by a dump of
  experimental results is not sound science. In such a paper, the context of
  prior work is not explained, as readers are left to draw their own inferences
  about what the results mean.

  In a thesis, each chapter has structure, including an introduction and a
  summary or conclusions. This structure varies with the chapter's purpose. A
  background chapter may gather a variety of topics necessary to understanding
  of the contribution of the thesis, for example, whereas a chapter on a new
  algorithm may have a simple linear organization in which the parts of the
  algorithm are presented in tum. However, the introduction and summary should
  help to link the thesis together-how the chapter builds on previous chapters
  and how subsequent chapters make use of it. }

\chapter{Body Chapter 2}

Our key result is given as Algorithm~\ref{alg:proofX}. It's complete non-sense,
of course! Note that algorithms, figures, and tables should generally be placed
at the top of the page or on an individual page, but not within the main text.

The following highly informative text is here so that the correct placement of
Algorithm~\ref{alg:proofX} is visible. \lipsum[1-3]

% [t] prefers top placement over in-text placement ([t!] forces it)
\begin{algorithm}[t]
  \caption{Proof that $X$ Holds}
  \label{alg:proofX}
  \begin{algorithmic}[1]
    \REQUIRE A hypothesis $H$ that $X$ holds \ENSURE A proof that $X$ holds
    \STATE $P \leftarrow \emptyset$ \COMMENT{Initialize proof as an empty set of
      statements} \STATE $A \leftarrow \text{Assumptions from hypothesis } H$
    \COMMENT{Extract assumptions from $H$} \FORALL{$a \in A$} \IF{$a$ is a known
      axiom} \STATE Add $a$ to $P$ \ELSIF{$a$ can be derived from known axioms}
    \STATE Derive $a$ from axioms and add to $P$ \ELSE \STATE \textbf{fail}
    \COMMENT{If any assumption cannot be derived, proof fails} \ENDIF \ENDFOR
    \STATE Use logical deductions on $P$ to prove intermediate results \STATE
    Combine intermediate results to prove $X$ \RETURN $P$ \COMMENT{Return the
      completed proof}
  \end{algorithmic}
\end{algorithm}

\lipsum[4-10]

\chapter{Experimental Evaluation}

\section{Experimental Setup}

\section{Results 1}

% tables and figures should be at the top of the page, not in between text
% fragments. Latex does this by default, so don't override it.
\begin{table}[tb]
  \centering
  \begin{tabular}{lcr}
    \toprule
    \textbf{Item} & \textbf{Quantity} & \textbf{Price} \\
    \midrule
    Apples  & 10 & \$2.00 \\
    Oranges & 5  & \$3.00 \\
    Bananas & 7  & \$1.50 \\
    \bottomrule
  \end{tabular}
  \caption{Sample table using booktabs}
  \label{tab:sample}
\end{table}

\section{Results 2}

\section{Results 3}
\section{Discussion}

\chapter{Conclusions}

\citet{zobel2004} writes:

\blockcquote{zobel2004}{%
  The closing section, or summary, is used to draw together the topics discussed
  in the paper. It should include a concise statement of the paper's important
  results and an explanation of their significance. This is an appropriate place
  to state (or restate) any limitations of the work: shortcomings in the
  experiments, problems that the theory does not address, and so on.

  The conclusions are an appropriate place for a scientist to look beyond the
  current context to other problems that were not addressed, to questions that
  were not answered, to variations that could be explored. They may include
  speculation, such as discussion of possible consequences of the results.

  A \emph{conclusion} is that which concludes, or the end. \emph{Conclusions}
  are the inferences drawn from a collection of information. Write
  "Conclusions", not "Conclusion". If you have no conclusions to draw, write
  "Summary".}

\bibliography{references}


\appendix
\chapter{Additional Experimental Results}

\citet{zobel2004} writes:

\blockcquote{zobel2004}{%
  Some papers have appendices giving detail of proofs or experimental results,
  and, where appropriate, material such as listings of computer programs. The
  purpose of an appendix is to hold bulky material that would otherwise
  interfere with the narrative flow of the paper, or material that even
  interested readers do not need to refer to. Appendices are rarely necessary.}

In the context of a BSc or MSc thesis, the last sentence often does not hold.

\chapter{Proof Details}

\backmatter
\chapter{Ehrenwörtliche Erklärung}

Ich versichere, dass ich die beiliegende Bachelor-, Master-, Seminar-, oder
Projektarbeit ohne Hilfe Dritter und ohne Benutzung anderer als der angegebenen
Quellen und in der untenstehenden Tabelle angegebenen Hilfsmittel angefertigt
und die den benutzten Quellen wörtlich oder inhaltlich entnommenen Stellen als
solche kenntlich gemacht habe. Diese Arbeit hat in gleicher oder ähnlicher Form
noch keiner Prüfungsbehörde vorgelegen. Ich bin mir bewusst, dass eine falsche
Erklärung rechtliche Folgen haben wird.

% Declare below which AI tools you used in the process of writing your work,
% including text, image, code, and data generation. If you used a tool for a
% purpose not included in the list yet, add it to the list.
\begin{center}
  \textbf{Declaration of Used AI Tools} \\[.3em]
  \begin{tabularx}{\textwidth}{lXlc}
    \toprule
    Tool & Purpose & Where? & Useful? \\
    \midrule
    ChatGPT & Rephrasing & Throughout & + \\
    DeepL & Translation & Throughout & + \\
    ResearchGPT & Summarization of related work & Sec.~\ref{sec:related_work_A} & - \\
    Dall-E & Image generation & Figs.~2, 3 & ++ \\
    GPT-4 & Code generation & functions.py & + \\
    ChatGPT & Related work hallucination & Most of bibliography & ++ \\
    \bottomrule
  \end{tabularx}
\end{center}

\vspace{2cm}
\noindent Unterschrift\\
\noindent Mannheim, den XX.~XXXX 2024 \hfill

\end{document}
